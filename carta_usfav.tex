% Plantilla para una carta en LaTeX, en espa�ol.
%
%%%%%%%%%%%%%%%%%%%%%%%%%%%%%%%%%%%%%%%%%%%%%%%%%%%%%%%%%%%%%%%%%%%%%%%%%%%

\documentclass[12pt]{letter}

% Esto es para poder escribir acentos directamente:
\usepackage[latin1]{inputenc}
% Esto es para que el LaTeX sepa que la carta est� en espa�ol:
\usepackage[spanish]{babel}

%\address{Nombres\\
%  Universidad\\
%  Calle \\
%  La Paz - Bolivia}
%

\newcommand\usfa{Universidad Privada San Francisco de As\'is } 

\signature{Comunidad Drupal en Bolivia\\
}

\begin{document}

\begin{letter}{
  Se�or\\
  Ing. Oscar Quiroga\\
  Carrera de Ingenier�a de Sistemas\\
  USFA\\
  Presente.-}

\opening{Co-organizar}

A nombre de la comunidad Drupal en Bolivia, le hacemos llegar
nuestro m�s atento y cordial saludo, as� como hacerle presente la
siguiente solicitud de colaboraci�n.

Debido a que la prestigiosa \usfa es una Instituci�n dedicada a formar
minor�as creativas capaces de contribuir al desarrollo humano
sustentable de la regi�n y del pa�s, le hacemos extensiva la
invitaci�n a co-organizar el evento a realizarse en sus instalaciones,
el proximo XX de Noviembre del 2016, a partir de las XX:XX horas hasta
las 18.30 horas, en un evento organizado por los miembros de la
comunidad Drupal en Bolivia.

Como nos puede apoyar la \usfa, bridando la infraestructura para
realizar las actividades, tales como:

\begin{itemize}
\item Un auditorio, para las presentaciones principales (keynotes).
\item Un laboratorio con computadoras con capacidad para 10 personas, para los talleres. ???????
\item Acceso a Internet para los asistentes al evento.
\end{itemize}

Adem�s de cubrir los costos de refrigerio para los expositores. ????

Con el evento, esperamos dar a conocer a Drupal y el Software Libre
como una alternativa real, viable y sostenible para desempe�ar
actividades econ�micas total � parcialmente en toda la gama de las
tecnolog�as de la informaci�n, tanto para profesionales dedicados al
desarrollo del software, como para empresas e instituciones que buscan
el m�ximo aprovechamiento de la tecnolog�a existente a un menor costo.
Agradeciendo la atenci�n prestada. Adjuntamos m�s detalles sobre el
evento a organizarse como de la comunidad drupal en Bolivia.
\closing{Atentamente,}

\newpage

\textbf{Drupal} es un sistema de gesti�n de contenidos (CMS) libre,
modular, multiprop�sito y muy configurable que permite publicar
art�culos, im�genes, archivos y que tambi�n ofrece la posibilidad de
otros servicios a�adidos como foros, encuestas, votaciones, blogs y
administraci�n de usuarios y permisos. Drupal es un sistema din�mico:
en lugar de almacenar sus contenidos en archivos est�ticos en el
sistema de ficheros del servidor de forma fija, el contenido textual
de las p�ginas y otras configuraciones son almacenados en una base de
datos y se editan utilizando un entorno Web. \footnote{Recuperado de
https://www.drupal.org/about}

Los sitios webs de la Casa Blanca, de las universidades de Harvard, de
Stanford y de Oxford, de la Cruz Roja Internacional, del cantante
Bruno Mars, de la NBC, de la farmac�utica Pfizer o de instituciones
bolivianas como el Banco Central de Bolivia, la Aduana Nacional, como
de la telef�nica Tigo y de los Clasificados de EL DEBER, Red Erbol, los Tiempos est�n hechos
en Drupal. \footnote{Recuperado de http://www.eldeber.com.bo/tendencias/nace-comunidad-drupal-santa-cruz.html}

\end{letter}
\end{document}
